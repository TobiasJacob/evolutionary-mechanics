\documentclass[12pt]{article}
\usepackage[utf8]{inputenc}
\usepackage{tikz}
\usepackage{amsmath}
\usepackage{graphics}
\usetikzlibrary{positioning}
\usepackage[margin=1in]{geometry}

\title{\vspace{-3em} Evolutionary Algorithms for Mechanical Structures}

\author{Tobias Jacob \\ Raffaele Galliera \\ Ali Muddasar}
\date{November 22, 2020}

\begin{document}
\maketitle

\section*{Project Report}
    During this first week of implementation, we worked on setting up the C++ project, implemented a framework of the project and a first running iteration.
\begin{itemize}
    \item Setup of the project, Catch2 testing framework, GitHub, Makefile, Trello and the general code structure; [All]
    \item Implemented the creation of the FEM equation system, the CG solving method, the sparse matrix implementation and performance evaluation; [Tobias Jacob]
    \item First iteration of the evolutionary optimizer, implementing general concept of the Genetic approach and reproduction system; [Raffaele Galliera]
    \item Implemented a Floodfilling algorithm, in order to make sure that all the planes generated stay connected together; [Muddasar Ali]
    \item We planned how we want to parallelize the equation solver and distribute the evolution and which parts of it will have a focus on. [All]
\end{itemize}

\section*{Speedups}

We want to solve an equation system of at least $n^2 \approx 80799$ equations. Possibly, even more. Solving time of an equation system grows in general as a cubic of equation size, so in our case $n^6$. We implemented the \textbf{CG method}, which reduces this time down to $O(n^4 i)$, with $i$ being the required iterations, and $n^4$ the time for the vector matrix multiplication with $n^2$ rows / columns. A big advantage is, that we have sparse matrices with typically 8 values per row. By extending our matrix implementation to a \textbf{sparse matrix implementation}, we improved the execution speed to $O(n^2 i)$, which had the biggest impact on performance.

Using \texttt{-Ofast} instead of \texttt{-O3} improved the speed by roughly 25.4\%. Using aligned memory did not improve the performance. Using a vector instead of a linked list for the matrix storage did not have a significant impact.

\section*{Profiling of the Simulation}

Simulating the mechanical structures is taking up the most time. For profiling, we created a $n \times n$ grid, added supports on the bottom, and a force on the top. Then we tried to calculate the stress on the structure. We measured the time for different parts of the solver.


\begin{itemize}
    \item A $n \times n$ grid has $2 (n + 1)^2 - 3$ degrees of freedom. The matrix we solve therefore has $O(n^2)$ equations.
    
    \item The estimated steps of the solution seems to be proportional to the maximum length between two points, so $i=O(n)$. An exact calculation is difficult because it is dependent on the condition of the matrix.
    
    \item For each degree of freedom the connections to the four neighbours have to be added to the equation. The equation setup time is therefore $O(n^2)$.
    
    \item The solving time is $O(n^2 c i)$, with $n^2$ being the number of equations, $c=8$ being the connections to different neighbour planes and $i$ being the number of required iterations. Since $c$ is constant, and $i = O(n)$, the total solving time is $O(n^3)$.
\end{itemize}

Currently, we are able to solve the equation for a $200\times200$ grid in roughly 22 seconds.

\begin{table}[h]
    \centering
    \small
    \begin{tabular}{rrrrrr}
        $n$ & Equations & Steps for solution & Equation setup time ($\mu s$) & Solving time ($\mu s$)\\
        - & $O(n^2)$ & $O(n)$ & $O(n^2)$ & $O(n^3)$\\
        200 & 80799 & 6555 & 98695 & 22344100 \\
        100 & 23187 & 2146 & 23187 & 1726910 \\
        50 & 5199 & 1039 & 6203 & 154818 \\
        25 & 1349 & 449 & 1626 & 17768 \\
        12 & 335 & 210 & 361 & 1786
    \end{tabular}
    \caption{Execution time for FEM solver}
    \label{tab:profiling}
\end{table}



\end{document}
